% Options for packages loaded elsewhere
\PassOptionsToPackage{unicode}{hyperref}
\PassOptionsToPackage{hyphens}{url}
%
\documentclass[
]{article}
\usepackage{amsmath,amssymb}
\usepackage{lmodern}
\usepackage{ifxetex,ifluatex}
\ifnum 0\ifxetex 1\fi\ifluatex 1\fi=0 % if pdftex
  \usepackage[T1]{fontenc}
  \usepackage[utf8]{inputenc}
  \usepackage{textcomp} % provide euro and other symbols
\else % if luatex or xetex
  \usepackage{unicode-math}
  \defaultfontfeatures{Scale=MatchLowercase}
  \defaultfontfeatures[\rmfamily]{Ligatures=TeX,Scale=1}
\fi
% Use upquote if available, for straight quotes in verbatim environments
\IfFileExists{upquote.sty}{\usepackage{upquote}}{}
\IfFileExists{microtype.sty}{% use microtype if available
  \usepackage[]{microtype}
  \UseMicrotypeSet[protrusion]{basicmath} % disable protrusion for tt fonts
}{}
\makeatletter
\@ifundefined{KOMAClassName}{% if non-KOMA class
  \IfFileExists{parskip.sty}{%
    \usepackage{parskip}
  }{% else
    \setlength{\parindent}{0pt}
    \setlength{\parskip}{6pt plus 2pt minus 1pt}}
}{% if KOMA class
  \KOMAoptions{parskip=half}}
\makeatother
\usepackage{xcolor}
\IfFileExists{xurl.sty}{\usepackage{xurl}}{} % add URL line breaks if available
\IfFileExists{bookmark.sty}{\usepackage{bookmark}}{\usepackage{hyperref}}
\hypersetup{
  hidelinks,
  pdfcreator={LaTeX via pandoc}}
\urlstyle{same} % disable monospaced font for URLs
\usepackage[margin=1in]{geometry}
\usepackage{color}
\usepackage{fancyvrb}
\newcommand{\VerbBar}{|}
\newcommand{\VERB}{\Verb[commandchars=\\\{\}]}
\DefineVerbatimEnvironment{Highlighting}{Verbatim}{commandchars=\\\{\}}
% Add ',fontsize=\small' for more characters per line
\usepackage{framed}
\definecolor{shadecolor}{RGB}{248,248,248}
\newenvironment{Shaded}{\begin{snugshade}}{\end{snugshade}}
\newcommand{\AlertTok}[1]{\textcolor[rgb]{0.94,0.16,0.16}{#1}}
\newcommand{\AnnotationTok}[1]{\textcolor[rgb]{0.56,0.35,0.01}{\textbf{\textit{#1}}}}
\newcommand{\AttributeTok}[1]{\textcolor[rgb]{0.77,0.63,0.00}{#1}}
\newcommand{\BaseNTok}[1]{\textcolor[rgb]{0.00,0.00,0.81}{#1}}
\newcommand{\BuiltInTok}[1]{#1}
\newcommand{\CharTok}[1]{\textcolor[rgb]{0.31,0.60,0.02}{#1}}
\newcommand{\CommentTok}[1]{\textcolor[rgb]{0.56,0.35,0.01}{\textit{#1}}}
\newcommand{\CommentVarTok}[1]{\textcolor[rgb]{0.56,0.35,0.01}{\textbf{\textit{#1}}}}
\newcommand{\ConstantTok}[1]{\textcolor[rgb]{0.00,0.00,0.00}{#1}}
\newcommand{\ControlFlowTok}[1]{\textcolor[rgb]{0.13,0.29,0.53}{\textbf{#1}}}
\newcommand{\DataTypeTok}[1]{\textcolor[rgb]{0.13,0.29,0.53}{#1}}
\newcommand{\DecValTok}[1]{\textcolor[rgb]{0.00,0.00,0.81}{#1}}
\newcommand{\DocumentationTok}[1]{\textcolor[rgb]{0.56,0.35,0.01}{\textbf{\textit{#1}}}}
\newcommand{\ErrorTok}[1]{\textcolor[rgb]{0.64,0.00,0.00}{\textbf{#1}}}
\newcommand{\ExtensionTok}[1]{#1}
\newcommand{\FloatTok}[1]{\textcolor[rgb]{0.00,0.00,0.81}{#1}}
\newcommand{\FunctionTok}[1]{\textcolor[rgb]{0.00,0.00,0.00}{#1}}
\newcommand{\ImportTok}[1]{#1}
\newcommand{\InformationTok}[1]{\textcolor[rgb]{0.56,0.35,0.01}{\textbf{\textit{#1}}}}
\newcommand{\KeywordTok}[1]{\textcolor[rgb]{0.13,0.29,0.53}{\textbf{#1}}}
\newcommand{\NormalTok}[1]{#1}
\newcommand{\OperatorTok}[1]{\textcolor[rgb]{0.81,0.36,0.00}{\textbf{#1}}}
\newcommand{\OtherTok}[1]{\textcolor[rgb]{0.56,0.35,0.01}{#1}}
\newcommand{\PreprocessorTok}[1]{\textcolor[rgb]{0.56,0.35,0.01}{\textit{#1}}}
\newcommand{\RegionMarkerTok}[1]{#1}
\newcommand{\SpecialCharTok}[1]{\textcolor[rgb]{0.00,0.00,0.00}{#1}}
\newcommand{\SpecialStringTok}[1]{\textcolor[rgb]{0.31,0.60,0.02}{#1}}
\newcommand{\StringTok}[1]{\textcolor[rgb]{0.31,0.60,0.02}{#1}}
\newcommand{\VariableTok}[1]{\textcolor[rgb]{0.00,0.00,0.00}{#1}}
\newcommand{\VerbatimStringTok}[1]{\textcolor[rgb]{0.31,0.60,0.02}{#1}}
\newcommand{\WarningTok}[1]{\textcolor[rgb]{0.56,0.35,0.01}{\textbf{\textit{#1}}}}
\usepackage{graphicx}
\makeatletter
\def\maxwidth{\ifdim\Gin@nat@width>\linewidth\linewidth\else\Gin@nat@width\fi}
\def\maxheight{\ifdim\Gin@nat@height>\textheight\textheight\else\Gin@nat@height\fi}
\makeatother
% Scale images if necessary, so that they will not overflow the page
% margins by default, and it is still possible to overwrite the defaults
% using explicit options in \includegraphics[width, height, ...]{}
\setkeys{Gin}{width=\maxwidth,height=\maxheight,keepaspectratio}
% Set default figure placement to htbp
\makeatletter
\def\fps@figure{htbp}
\makeatother
\setlength{\emergencystretch}{3em} % prevent overfull lines
\providecommand{\tightlist}{%
  \setlength{\itemsep}{0pt}\setlength{\parskip}{0pt}}
\setcounter{secnumdepth}{-\maxdimen} % remove section numbering
\ifluatex
  \usepackage{selnolig}  % disable illegal ligatures
\fi

\author{}
\date{\vspace{-2.5em}}

\begin{document}

\begin{Shaded}
\begin{Highlighting}[]
\FunctionTok{library}\NormalTok{(tidyverse)}
\end{Highlighting}
\end{Shaded}

\begin{verbatim}
## -- Attaching packages --------------------------------------- tidyverse 1.3.1 --
\end{verbatim}

\begin{verbatim}
## v ggplot2 3.3.5     v purrr   0.3.4
## v tibble  3.1.2     v dplyr   1.0.7
## v tidyr   1.1.3     v stringr 1.4.0
## v readr   1.4.0     v forcats 0.5.1
\end{verbatim}

\begin{verbatim}
## -- Conflicts ------------------------------------------ tidyverse_conflicts() --
## x dplyr::filter() masks stats::filter()
## x dplyr::lag()    masks stats::lag()
\end{verbatim}

\begin{Shaded}
\begin{Highlighting}[]
\NormalTok{games }\OtherTok{\textless{}{-}} \FunctionTok{read\_csv}\NormalTok{(}\StringTok{"https://raw.githubusercontent.com/rfordatascience/tidytuesday/master/data/2019/2019{-}07{-}30/video\_games.csv"}\NormalTok{)}
\end{Highlighting}
\end{Shaded}

\begin{verbatim}
## 
## -- Column specification --------------------------------------------------------
## cols(
##   number = col_double(),
##   game = col_character(),
##   release_date = col_character(),
##   price = col_double(),
##   owners = col_character(),
##   developer = col_character(),
##   publisher = col_character(),
##   average_playtime = col_double(),
##   median_playtime = col_double(),
##   metascore = col_double()
## )
\end{verbatim}

\begin{Shaded}
\begin{Highlighting}[]
\NormalTok{games}
\end{Highlighting}
\end{Shaded}

\begin{verbatim}
## # A tibble: 26,688 x 10
##    number game    release_date price owners developer publisher average_playtime
##     <dbl> <chr>   <chr>        <dbl> <chr>  <chr>     <chr>                <dbl>
##  1      1 Half-L~ Nov 16, 2004  9.99 10,00~ Valve     Valve                  110
##  2      3 Counte~ Nov 1, 2004   9.99 10,00~ Valve     Valve                  236
##  3     21 Counte~ Mar 1, 2004   9.99 10,00~ Valve     Valve                   10
##  4     47 Half-L~ Nov 1, 2004   4.99 5,000~ Valve     Valve                    0
##  5     36 Half-L~ Jun 1, 2004   9.99 2,000~ Valve     Valve                    0
##  6     52 CS2D    Dec 24, 2004 NA    1,000~ Unreal S~ Unreal S~               16
##  7      2 Unreal~ Mar 16, 2004 15.0  500,0~ Epic Gam~ Epic Gam~                0
##  8      4 DOOM 3  Aug 3, 2004   4.99 500,0~ id Softw~ id Softw~                0
##  9     14 Beyond~ Apr 27, 2004  5.99 500,0~ Larian S~ Larian S~                0
## 10     40 Hitman~ Apr 20, 2004  8.99 500,0~ Io-Inter~ Io-Inter~                0
## # ... with 26,678 more rows, and 2 more variables: median_playtime <dbl>,
## #   metascore <dbl>
\end{verbatim}

\begin{Shaded}
\begin{Highlighting}[]
\NormalTok{games }\OtherTok{\textless{}{-}}\NormalTok{ games }\SpecialCharTok{\%\textgreater{}\%}
  \FunctionTok{mutate}\NormalTok{(}\AttributeTok{meta\_rating\_class =} \FunctionTok{case\_when}\NormalTok{(metascore }\SpecialCharTok{\textgreater{}=} \DecValTok{90} \SpecialCharTok{\textasciitilde{}} \StringTok{"Universal Acclaim"}\NormalTok{,}
\NormalTok{                                       metascore }\SpecialCharTok{\textgreater{}=} \DecValTok{75} \SpecialCharTok{\textasciitilde{}} \StringTok{"Generally Favorable"}\NormalTok{,}
\NormalTok{                                       metascore }\SpecialCharTok{\textgreater{}=} \DecValTok{50} \SpecialCharTok{\textasciitilde{}} \StringTok{"Mixed/Average"}\NormalTok{,}
\NormalTok{                                       metascore }\SpecialCharTok{\textgreater{}=} \DecValTok{20} \SpecialCharTok{\textasciitilde{}} \StringTok{"Generally Unfavorable"}\NormalTok{,}
\NormalTok{                                       metascore }\SpecialCharTok{\textless{}} \DecValTok{20} \SpecialCharTok{\textasciitilde{}} \StringTok{"Overwhelming Dislike"}\NormalTok{))}
\NormalTok{games}
\end{Highlighting}
\end{Shaded}

\begin{verbatim}
## # A tibble: 26,688 x 11
##    number game    release_date price owners developer publisher average_playtime
##     <dbl> <chr>   <chr>        <dbl> <chr>  <chr>     <chr>                <dbl>
##  1      1 Half-L~ Nov 16, 2004  9.99 10,00~ Valve     Valve                  110
##  2      3 Counte~ Nov 1, 2004   9.99 10,00~ Valve     Valve                  236
##  3     21 Counte~ Mar 1, 2004   9.99 10,00~ Valve     Valve                   10
##  4     47 Half-L~ Nov 1, 2004   4.99 5,000~ Valve     Valve                    0
##  5     36 Half-L~ Jun 1, 2004   9.99 2,000~ Valve     Valve                    0
##  6     52 CS2D    Dec 24, 2004 NA    1,000~ Unreal S~ Unreal S~               16
##  7      2 Unreal~ Mar 16, 2004 15.0  500,0~ Epic Gam~ Epic Gam~                0
##  8      4 DOOM 3  Aug 3, 2004   4.99 500,0~ id Softw~ id Softw~                0
##  9     14 Beyond~ Apr 27, 2004  5.99 500,0~ Larian S~ Larian S~                0
## 10     40 Hitman~ Apr 20, 2004  8.99 500,0~ Io-Inter~ Io-Inter~                0
## # ... with 26,678 more rows, and 3 more variables: median_playtime <dbl>,
## #   metascore <dbl>, meta_rating_class <chr>
\end{verbatim}

\hypertarget{data-exploration}{%
\section{Data Exploration}\label{data-exploration}}

A summary of our dataset:

\begin{Shaded}
\begin{Highlighting}[]
\FunctionTok{summary}\NormalTok{(games)}
\end{Highlighting}
\end{Shaded}

\begin{verbatim}
##      number         game           release_date           price        
##  Min.   :   1   Length:26688       Length:26688       Min.   :  0.490  
##  1st Qu.: 821   Class :character   Class :character   1st Qu.:  2.990  
##  Median :2356   Mode  :character   Mode  :character   Median :  5.990  
##  Mean   :2904                                         Mean   :  8.947  
##  3rd Qu.:4523                                         3rd Qu.:  9.990  
##  Max.   :8846                                         Max.   :595.990  
##                                                       NA's   :3095     
##     owners           developer          publisher         average_playtime  
##  Length:26688       Length:26688       Length:26688       Min.   :   0.000  
##  Class :character   Class :character   Class :character   1st Qu.:   0.000  
##  Mode  :character   Mode  :character   Mode  :character   Median :   0.000  
##                                                           Mean   :   9.057  
##                                                           3rd Qu.:   0.000  
##                                                           Max.   :5670.000  
##                                                           NA's   :9         
##  median_playtime     metascore     meta_rating_class 
##  Min.   :   0.00   Min.   :20.00   Length:26688      
##  1st Qu.:   0.00   1st Qu.:66.00   Class :character  
##  Median :   0.00   Median :73.00   Mode  :character  
##  Mean   :   5.16   Mean   :71.89                     
##  3rd Qu.:   0.00   3rd Qu.:80.00                     
##  Max.   :3293.00   Max.   :98.00                     
##  NA's   :12        NA's   :23838
\end{verbatim}

\hypertarget{univariate-eda}{%
\section{Univariate EDA}\label{univariate-eda}}

First, before we get into answering our questions, we do a little bit of
univariate exploratory data anylsis to visualize some of our data.

\hypertarget{metascore}{%
\subsection{Metascore}\label{metascore}}

CHECK TO SEE IF THIS IS GOOD

NEW:

The \href{https://www.metacritic.com/}{Metacritic} website is a popular
website that takes many critic reviews for media including video games.
Metacritic reviews is one way to tell if a game would be worth buying or
``fun''. Our data set includes a column for the Metacritic score's that
are available on the website, and earlier we created a new column to
include the classifications that Metacritic uses. The levels of rating
can be from 1 to 100, and the classifications from overwhelming dislike
to universal acclaim. Here, we create a bar blot to show number of games
with certain scores, including filling games by their classification.

ORIGINAL:

One of the most Important parts of a game is how fun it is and whether
or not it is considered a ``good game.'' One of the most popular ways to
do this is to look at the Meta Score for the game, made by the popular
rating website \href{https://www.metacritic.com/}{Metacritic} This
website has ratings for several categories of media, but we are going to
be using the ``Game'' category. Inside this category, there are
different levels of rating from 1-100, with the colors indicating

\begin{itemize}
\tightlist
\item
  100-90 = \emph{Universally Acclaimed}
\item
  89-75 = \emph{Generally Favorable}
\item
  74-50 = \emph{Mixed/Average}
\item
  49-1 = \emph{Generally Unfavorable}
\end{itemize}

\begin{Shaded}
\begin{Highlighting}[]
\NormalTok{games }\SpecialCharTok{\%\textgreater{}\%}
  \FunctionTok{ggplot}\NormalTok{() }\SpecialCharTok{+}
  \FunctionTok{geom\_bar}\NormalTok{(}\FunctionTok{aes}\NormalTok{(}\AttributeTok{x =}\NormalTok{ metascore,}
               \AttributeTok{fill =}\NormalTok{ meta\_rating\_class))}
\end{Highlighting}
\end{Shaded}

\begin{verbatim}
## Warning: Removed 23838 rows containing non-finite values (stat_count).
\end{verbatim}

\includegraphics{EDA_files/figure-latex/unnamed-chunk-6-1.pdf}

As we can see from the bar blot, most of the games tend to be in
mixed/average or generally favorable, with very few games being
generally unfavorable or universal acclaim.

\hypertarget{price}{%
\subsection{Price}\label{price}}

An important part about a video game would be it's price. Likely, games
that are larger and more expensive to make will be priced higher. Here,
we visualize the data distribution on price. First, we have decided to
omit games with a price greater than 400, because there is only one game
with no average\_playtime, median\_playtime, or metascore. There will be
two different histograms made, one including prices less than 400 to
show a greater scale of prices and the other with games priced less than
100, to show where a larger portion of the games are priced, as most
games are less than 100 dollars.

\begin{Shaded}
\begin{Highlighting}[]
\NormalTok{games }\SpecialCharTok{\%\textgreater{}\%}
  \FunctionTok{filter}\NormalTok{(price }\SpecialCharTok{\textgreater{}} \DecValTok{400}\NormalTok{)}
\end{Highlighting}
\end{Shaded}

\begin{verbatim}
## # A tibble: 1 x 11
##   number game   release_date price owners  developer  publisher average_playtime
##    <dbl> <chr>  <chr>        <dbl> <chr>   <chr>      <chr>                <dbl>
## 1   2958 ADR-L~ Apr 12, 2018  596. 0 .. 2~ Suomen Ku~ Suomen K~                0
## # ... with 3 more variables: median_playtime <dbl>, metascore <dbl>,
## #   meta_rating_class <chr>
\end{verbatim}

Games with prices less than 400:

\begin{Shaded}
\begin{Highlighting}[]
\NormalTok{games }\SpecialCharTok{\%\textgreater{}\%}
  \FunctionTok{filter}\NormalTok{(price }\SpecialCharTok{\textless{}} \DecValTok{400}\NormalTok{) }\SpecialCharTok{\%\textgreater{}\%}
  \FunctionTok{ggplot}\NormalTok{() }\SpecialCharTok{+}
  \FunctionTok{geom\_histogram}\NormalTok{(}
    \AttributeTok{mapping =} \FunctionTok{aes}\NormalTok{(}
      \AttributeTok{x =}\NormalTok{ price),}
    \AttributeTok{binwidth =}  \DecValTok{5}
\NormalTok{  ) }\SpecialCharTok{+}
  \FunctionTok{labs}\NormalTok{(}
    \AttributeTok{title =} \StringTok{"Games with price less than 400"}\NormalTok{,}
    \AttributeTok{x =} \StringTok{"Price of Games"}\NormalTok{,}
    \AttributeTok{y =} \StringTok{"Count"}
\NormalTok{  )}
\end{Highlighting}
\end{Shaded}

\includegraphics{EDA_files/figure-latex/unnamed-chunk-8-1.pdf}

Games with prices less than 100:

\begin{Shaded}
\begin{Highlighting}[]
\NormalTok{games }\SpecialCharTok{\%\textgreater{}\%}
  \FunctionTok{filter}\NormalTok{(price }\SpecialCharTok{\textless{}} \DecValTok{100}\NormalTok{) }\SpecialCharTok{\%\textgreater{}\%}
  \FunctionTok{ggplot}\NormalTok{() }\SpecialCharTok{+}
  \FunctionTok{geom\_histogram}\NormalTok{(}
    \AttributeTok{mapping =} \FunctionTok{aes}\NormalTok{(}
      \AttributeTok{x =}\NormalTok{ price),}
    \AttributeTok{binwidth =} \DecValTok{5}
\NormalTok{  ) }\SpecialCharTok{+}
    \FunctionTok{labs}\NormalTok{(}
    \AttributeTok{title =} \StringTok{"Games with price less than 100"}\NormalTok{,}
    \AttributeTok{x =} \StringTok{"Price of Games"}\NormalTok{,}
    \AttributeTok{y =} \StringTok{"Count"}
\NormalTok{  )}
\end{Highlighting}
\end{Shaded}

\includegraphics{EDA_files/figure-latex/unnamed-chunk-9-1.pdf}

\hypertarget{developer}{%
\subsection{Developer}\label{developer}}

Here, we look at the distribution of games by developer:

\begin{Shaded}
\begin{Highlighting}[]
\NormalTok{games }\SpecialCharTok{\%\textgreater{}\%}
  \FunctionTok{drop\_na}\NormalTok{() }\SpecialCharTok{\%\textgreater{}\%}
  \FunctionTok{group\_by}\NormalTok{(developer) }\SpecialCharTok{\%\textgreater{}\%}
  \FunctionTok{summarize}\NormalTok{(}\AttributeTok{count =} \FunctionTok{n}\NormalTok{()) }\SpecialCharTok{\%\textgreater{}\%}
  \FunctionTok{arrange}\NormalTok{(}\FunctionTok{desc}\NormalTok{(count)) }\SpecialCharTok{\%\textgreater{}\%}
  \FunctionTok{slice}\NormalTok{(}\DecValTok{1}\SpecialCharTok{:}\DecValTok{10}\NormalTok{) }\SpecialCharTok{\%\textgreater{}\%}
  \FunctionTok{ggplot}\NormalTok{() }\SpecialCharTok{+}
  \FunctionTok{geom\_col}\NormalTok{(}
    \AttributeTok{mapping =} \FunctionTok{aes}\NormalTok{(}
      \AttributeTok{x =} \FunctionTok{reorder}\NormalTok{(developer, count),}
      \AttributeTok{y =}\NormalTok{ count,}
      \AttributeTok{fill =}\NormalTok{ developer}
\NormalTok{    )}
\NormalTok{  ) }\SpecialCharTok{+}
  \FunctionTok{labs}\NormalTok{(}
    \AttributeTok{title =} \StringTok{"Distribution of Developers (Top 10)"}\NormalTok{,}
    \AttributeTok{x =} \StringTok{"Developer"}\NormalTok{,}
    \AttributeTok{y =} \StringTok{"Count"}
\NormalTok{  ) }\SpecialCharTok{+}
  \FunctionTok{coord\_flip}\NormalTok{()}
\end{Highlighting}
\end{Shaded}

\includegraphics{EDA_files/figure-latex/unnamed-chunk-10-1.pdf} One
quick observation we can make is that Ubisoft Montreal develops a lot of
games.

\hypertarget{publishers}{%
\subsection{Publishers}\label{publishers}}

We also look at the distribution for publishers.

\begin{Shaded}
\begin{Highlighting}[]
\NormalTok{games }\SpecialCharTok{\%\textgreater{}\%}
  \FunctionTok{drop\_na}\NormalTok{() }\SpecialCharTok{\%\textgreater{}\%}
  \FunctionTok{group\_by}\NormalTok{(publisher) }\SpecialCharTok{\%\textgreater{}\%}
  \FunctionTok{summarize}\NormalTok{(}\AttributeTok{count =} \FunctionTok{n}\NormalTok{()) }\SpecialCharTok{\%\textgreater{}\%}
  \FunctionTok{arrange}\NormalTok{(}\FunctionTok{desc}\NormalTok{(count)) }\SpecialCharTok{\%\textgreater{}\%}
  \FunctionTok{slice}\NormalTok{(}\DecValTok{1}\SpecialCharTok{:}\DecValTok{10}\NormalTok{) }\SpecialCharTok{\%\textgreater{}\%}
  \FunctionTok{ggplot}\NormalTok{() }\SpecialCharTok{+}
  \FunctionTok{geom\_col}\NormalTok{(}
    \AttributeTok{mapping =} \FunctionTok{aes}\NormalTok{(}
      \AttributeTok{x =} \FunctionTok{reorder}\NormalTok{(publisher, count),}
      \AttributeTok{y =}\NormalTok{ count,}
      \AttributeTok{fill =}\NormalTok{ publisher}
\NormalTok{    )}
\NormalTok{  ) }\SpecialCharTok{+}
  \FunctionTok{labs}\NormalTok{(}
    \AttributeTok{title =} \StringTok{"Distribution of Publisher\textquotesingle{}s (Top 10)"}\NormalTok{,}
    \AttributeTok{x =} \StringTok{"Publisher"}\NormalTok{,}
    \AttributeTok{y =} \StringTok{"Count"}
\NormalTok{  ) }\SpecialCharTok{+}
  \FunctionTok{coord\_flip}\NormalTok{()}
\end{Highlighting}
\end{Shaded}

\includegraphics{EDA_files/figure-latex/unnamed-chunk-11-1.pdf}

As we can see, Ubisoft stays on top for both number of games developed
and published.

\hypertarget{average-playtime}{%
\subsection{Average Playtime}\label{average-playtime}}

ASK ABOUT AVERAGE PLAYTIME/MEDIAN PLAYTIME

Here, we create a distribution by average playtime.

\begin{Shaded}
\begin{Highlighting}[]
\NormalTok{games }\SpecialCharTok{\%\textgreater{}\%}
  \FunctionTok{filter}\NormalTok{(average\_playtime }\SpecialCharTok{\textgreater{}} \DecValTok{0}\NormalTok{, average\_playtime }\SpecialCharTok{\textless{}=} \DecValTok{100}\NormalTok{) }\SpecialCharTok{\%\textgreater{}\%}
  \FunctionTok{drop\_na}\NormalTok{() }\SpecialCharTok{\%\textgreater{}\%}
  \FunctionTok{ggplot}\NormalTok{() }\SpecialCharTok{+}
  \FunctionTok{geom\_bar}\NormalTok{(}\FunctionTok{aes}\NormalTok{(}\AttributeTok{x =}\NormalTok{ average\_playtime))}
\end{Highlighting}
\end{Shaded}

\includegraphics{EDA_files/figure-latex/unnamed-chunk-12-1.pdf}

\begin{Shaded}
\begin{Highlighting}[]
\NormalTok{games }\SpecialCharTok{\%\textgreater{}\%}
  \FunctionTok{drop\_na}\NormalTok{() }\SpecialCharTok{\%\textgreater{}\%}
  \FunctionTok{group\_by}\NormalTok{(average\_playtime) }\SpecialCharTok{\%\textgreater{}\%}
  \FunctionTok{summarize}\NormalTok{(}\AttributeTok{count =} \FunctionTok{n}\NormalTok{()) }\SpecialCharTok{\%\textgreater{}\%}
  \FunctionTok{filter}\NormalTok{(average\_playtime }\SpecialCharTok{\textgreater{}} \DecValTok{0}\NormalTok{) }\SpecialCharTok{\%\textgreater{}\%}
  \FunctionTok{arrange}\NormalTok{(}\FunctionTok{desc}\NormalTok{(count)) }\SpecialCharTok{\%\textgreater{}\%}
  \FunctionTok{slice}\NormalTok{(}\DecValTok{1}\SpecialCharTok{:}\DecValTok{10}\NormalTok{) }\SpecialCharTok{\%\textgreater{}\%}
  \FunctionTok{ggplot}\NormalTok{() }\SpecialCharTok{+}
  \FunctionTok{geom\_col}\NormalTok{(}
    \AttributeTok{mapping =} \FunctionTok{aes}\NormalTok{(}
      \AttributeTok{x =}\NormalTok{ average\_playtime,}
      \AttributeTok{y =}\NormalTok{ count}
\NormalTok{    )}
\NormalTok{  ) }
\end{Highlighting}
\end{Shaded}

\includegraphics{EDA_files/figure-latex/unnamed-chunk-13-1.pdf}

\hypertarget{multivariatebivariate-exploratory-data-analysis}{%
\section{Multivariate/Bivariate Exploratory Data
Analysis}\label{multivariatebivariate-exploratory-data-analysis}}

Here, we did some multivariate exploration to find the relationship
between price and metascore. However, we had to use the slice function
to reduce the amount of games being used to lessen clutter.

\begin{Shaded}
\begin{Highlighting}[]
\NormalTok{games }\SpecialCharTok{\%\textgreater{}\%} 
  \FunctionTok{drop\_na}\NormalTok{() }\SpecialCharTok{\%\textgreater{}\%}
  \FunctionTok{slice}\NormalTok{(}\DecValTok{1}\SpecialCharTok{:}\DecValTok{300}\NormalTok{) }\SpecialCharTok{\%\textgreater{}\%}
  \FunctionTok{ggplot}\NormalTok{() }\SpecialCharTok{+}
  \FunctionTok{geom\_point}\NormalTok{(}
    \AttributeTok{mapping =} \FunctionTok{aes}\NormalTok{(}
      \AttributeTok{x =}\NormalTok{ metascore, }
      \AttributeTok{y =}\NormalTok{ price, }
      \AttributeTok{color =}\NormalTok{ meta\_rating\_class),}
    \AttributeTok{na.rm =} \ConstantTok{TRUE}\NormalTok{,}
    \AttributeTok{alpha =} \FloatTok{0.8}\NormalTok{) }\SpecialCharTok{+}
  \FunctionTok{labs}\NormalTok{(}
    \AttributeTok{title =} \StringTok{"Price by Metascore"}\NormalTok{, }
    \AttributeTok{x =} \StringTok{"Metascore"}\NormalTok{,}
    \AttributeTok{y =} \StringTok{"Price"}\NormalTok{,}
    \AttributeTok{color =} \StringTok{"Metascore Ranking"}\NormalTok{) }\SpecialCharTok{+}
    \FunctionTok{coord\_flip}\NormalTok{()}
\end{Highlighting}
\end{Shaded}

\includegraphics{EDA_files/figure-latex/unnamed-chunk-14-1.pdf}
Surprisingly, metascore seems to have almost no relationship with price,
as even as you look into a price of 10, a large amount are in 3
different categories.

\hypertarget{data-analysis}{%
\section{Data Analysis}\label{data-analysis}}

\hypertarget{how-does-price-affect-metacritic-score}{%
\subsection{How does price affect metacritic
score?}\label{how-does-price-affect-metacritic-score}}

To answer this question, we analyzed our data using a scatter plot
visualization. We were easily able to identify the distributions and
differences between each observation to find if metacritic scores really
had an affect on the price of the game.

\begin{Shaded}
\begin{Highlighting}[]
\NormalTok{games }\SpecialCharTok{\%\textgreater{}\%} 
  \FunctionTok{drop\_na}\NormalTok{() }\SpecialCharTok{\%\textgreater{}\%}
  \FunctionTok{slice}\NormalTok{(}\DecValTok{1}\SpecialCharTok{:}\DecValTok{300}\NormalTok{) }\SpecialCharTok{\%\textgreater{}\%}
  \FunctionTok{ggplot}\NormalTok{() }\SpecialCharTok{+}
  \FunctionTok{geom\_point}\NormalTok{(}
    \AttributeTok{mapping =} \FunctionTok{aes}\NormalTok{(}
      \AttributeTok{x =}\NormalTok{ metascore, }
      \AttributeTok{y =}\NormalTok{ price, }
      \AttributeTok{color =}\NormalTok{ meta\_rating\_class),}
    \AttributeTok{na.rm =} \ConstantTok{TRUE}\NormalTok{,}
    \AttributeTok{alpha =} \FloatTok{0.8}\NormalTok{) }\SpecialCharTok{+}
  \FunctionTok{labs}\NormalTok{(}
    \AttributeTok{title =} \StringTok{"Price by Metascore"}\NormalTok{, }
    \AttributeTok{x =} \StringTok{"Metascore"}\NormalTok{,}
    \AttributeTok{y =} \StringTok{"Price"}\NormalTok{,}
    \AttributeTok{color =} \StringTok{"Metascore Ranking"}\NormalTok{) }\SpecialCharTok{+}
    \FunctionTok{coord\_flip}\NormalTok{()}
\end{Highlighting}
\end{Shaded}

\includegraphics{EDA_files/figure-latex/unnamed-chunk-15-1.pdf}

What we see from this sample size of 500 observations, we can conclude
that the games with lower metacritic scores have lower prices.

\hypertarget{conclusion}{%
\section{Conclusion}\label{conclusion}}

\end{document}
